% This must be in the first 5 lines to tell arXiv to use pdfLaTeX, which is strongly recommended.
\pdfoutput=1
% In particular, the hyperref package requires pdfLaTeX in order to break URLs across lines.

\documentclass[11pt]{article}

% Change "review" to "final" to generate the final (sometimes called camera-ready) version.
% Change to "preprint" to generate a non-anonymous version with page numbers.
\usepackage[preprint]{acl}

% Standard package includes
\usepackage{times}
\usepackage{latexsym}

% For proper rendering and hyphenation of words containing Latin characters (including in bib files)
\usepackage[T1]{fontenc}
% For Vietnamese characters
% \usepackage[T5]{fontenc}
% See https://www.latex-project.org/help/documentation/encguide.pdf for other character sets

% This assumes your files are encoded as UTF8
\usepackage[utf8]{inputenc}

% This is not strictly necessary, and may be commented out,
% but it will improve the layout of the manuscript,
% and will typically save some space.
\usepackage{microtype}

% This is also not strictly necessary, and may be commented out.
% However, it will improve the aesthetics of text in
% the typewriter font.
\usepackage{inconsolata}

%Including images in your LaTeX document requires adding
%additional package(s)
\usepackage{graphicx}

% If the title and author information does not fit in the area allocated, uncomment the following
%
%\setlength\titlebox{<dim>}
%
% and set <dim> to something 5cm or larger.

\title{Grounding ``grounding'': How is grounding used within various AI conferences?}

\author{Eric Huang \\
  University of Waterloo\\
  \texttt{e48huang@uwaterloo.ca}}

%\author{
%  \textbf{First Author\textsuperscript{1}},
%  \textbf{Second Author\textsuperscript{1,2}},
%  \textbf{Third T. Author\textsuperscript{1}},
%  \textbf{Fourth Author\textsuperscript{1}},
%\\
%  \textbf{Fifth Author\textsuperscript{1,2}},
%  \textbf{Sixth Author\textsuperscript{1}},
%  \textbf{Seventh Author\textsuperscript{1}},
%  \textbf{Eighth Author \textsuperscript{1,2,3,4}},
%\\
%  \textbf{Ninth Author\textsuperscript{1}},
%  \textbf{Tenth Author\textsuperscript{1}},
%  \textbf{Eleventh E. Author\textsuperscript{1,2,3,4,5}},
%  \textbf{Twelfth Author\textsuperscript{1}},
%\\
%  \textbf{Thirteenth Author\textsuperscript{3}},
%  \textbf{Fourteenth F. Author\textsuperscript{2,4}},
%  \textbf{Fifteenth Author\textsuperscript{1}},
%  \textbf{Sixteenth Author\textsuperscript{1}},
%\\
%  \textbf{Seventeenth S. Author\textsuperscript{4,5}},
%  \textbf{Eighteenth Author\textsuperscript{3,4}},
%  \textbf{Nineteenth N. Author\textsuperscript{2,5}},
%  \textbf{Twentieth Author\textsuperscript{1}}
%\\
%\\
%  \textsuperscript{1}Affiliation 1,
%  \textsuperscript{2}Affiliation 2,
%  \textsuperscript{3}Affiliation 3,
%  \textsuperscript{4}Affiliation 4,
%  \textsuperscript{5}Affiliation 5
%\\
%  \small{
%    \textbf{Correspondence:} \href{mailto:email@domain}{email@domain}
%  }
%}

\begin{document}
\maketitle
\begin{abstract}
Terminology used within linguistics and AI conferences tend to be overused, leading to ambiguous meaning and difficulty navigating new papers. This paper will elucidate the various senses of the word ``grounding'' through qualitative analysis and deeper quantitative analysis of its various uses. This work will showcase how ``grounding'' is an overloaded term and guide users to understand how to more easily decipher and understand papers using the term. All code can be found at \url{https://git.uwaterloo.ca/e48huang/cs-784/-/tree/final_project/final_project?ref_type=heads} where a University of Waterloo account is required.
\end{abstract}

\section{Introduction}
Many conferences centering around Artificial Intelligence have existed for many decades, evolving over time on the types of problems that they tackle. While these problems change over time, so do the terminology, which have a tendency to evolve semantically, leading to overloaded terms. One such term is ``grounding'', the idea that one wishes to ensure that there is understanding or a common ground \cite{nakano-etal-2003-towards}. While this term seems simple, it is used in many various contexts, all of which requires different datasets, methods and metrics to evaluate, while being applied in different settings.

To better understand the term ``grounding'' and its usage, we perform both quantitative analysis and qualitative analysis. This paper explores the ``Seed42Lab/AI-paper-crawl'' HuggingFace dataset \cite{ai-paper-crawl} which collects full-text papers from 11 different conferences spanning from the first year of the conference to 2024. To first select different senses of the word ``grounding'', we perform preliminary quantitative analysis to filter for papers to further investigate. From these selected papers, we identify 9 related but distinct meanings of the word ``grounding''. We perform some literature review to understand how these different senses are understood, from its various datasets, methods, metrics and applications. Finally, for each of these word senses, we investigate how they have evolved over time.

\section{Paper Selection}
A simple search over the number of papers which have the term ``grounding'' quickly shows that it is infeasible to cover all possible instances. For example, the Association for Computational Linguistics (ACL) alone has 632 unique papers that have an instance of ``grounding'' (see Table~\ref{tab:counts}). While not all these instances are due to the paper itself being related to grounding, as they can simply include it within its bibliography, they are indicative that some filtering is necessary.

\begin{table}
  \centering
  \begin{tabular}{lc}
    \hline
    \textbf{Conference} & \textbf{Paper Count} \\
    \hline
    {AAAI}     & {772}           \\
    {ACL}     & {632}           \\
    {CVPR}     & {862}           \\
    {ECCV}     & {511}           \\
    {EMNLP}     & {575}           \\
    {ICCV}     & {341}           \\
    {ICLR}     & {360}           \\
    {ICML}     & {360}           \\
    {IJCAI}     & {654}           \\
    {NAACL}     & {226}           \\
    {NeurIPS}     & {654}           \\\hline
  \end{tabular}
  \caption{Counts of unique papers with ``grounding'' by conference found in the corpora.}
  \label{tab:counts}
\end{table}

To filter through these papers, we propose a method which selects the most relevant papers within a conference to the word ``grounding''. We take a naive approach where we select the top 10\% of papers with the word ``grounding''. We determine which papers are more important to ``grounding'' based on the word frequency, if ``grounding'' appears more often compared to other words within a paper, then it should be more relevant. This process (while ensuring uniqueness across conferences) resulted in 46 curated papers\footnote{\url{https://git.uwaterloo.ca/e48huang/cs-784/-/blob/e09a1c22c0de7e331ca16109a5f32b226dc6d9c5/final_project/grounding_top_p.txt}}, spanning from the years 2000 to 2024, shown in the following figure. While this selection of papers may not cover the breadth of senses of grounding might entail, as it misses on papers from the 1980's to 2000's, it does cover the most commonly used senses of the word.

\begin{figure}[t]
  \includegraphics[width=\columnwidth]{years_histogram.png}
  \caption{Count of selected conference papers per year}
  \label{fig:conference_years}
\end{figure}

\section{Grounding ``grounding''}
In this section we cover the different senses of the word ``grounding'' found in the 46 papers covered in the previous section. We will explore the common methods, datasets, applications and metrics revolving around each word. We will continue to elucidate the meanings of the word and its trend within the next section. The following sections also denote its subcategories of ``grounding'', however these subcategories are loosely defined and are chosen not necessarily based on modality, but the end 

\subsection{Visual Grounding}
Visual grounding, also known as image, phrase or referring expression grounding \cite{xiao2024visualgroundingsurvey, li-etal-2024-groundinggpt,ma2020learninggenerategroundedvisual,Islam_Gladstone_Iqbal_2023_PATRON, jiang-etal-2019-tiger,lu-etal-2022-extending, dou-peng-2021-improving,surikuchi-etal-2023-groovist} refers to the challenge of trying to localize specific regions within an image based on some textual description\footnote{other terms include natural language object retrieval or phrase localization \cite{gigapixel}}. Traditionally, this involved finding the phrase's referring region by predicting a bounding box around said region. As time has gone on however, there has been more and more types of challenges that one could tackle within visual grounding \cite{xiao2024visualgroundingsurvey}. From the 46 papers we filtered for, we have discovered the following subcategories involved in visual grounding:

\subsubsection{Classical Visual Grounding}
This entails the traditional problem of trying to predict a bounding box around a region \cite{li-etal-2024-groundinggpt,huang2021deconfoundedvisualgrounding,peng2023kosmos2groundingmultimodallarge}. Recent papers have found different methods in an attempt to improve performance.  \citet{Zeng_2024_CVPR_Comp_Challenge} improves compositional understanding through providing a harder dataset that relies on introducing different compositions of objects. \citet{neurips_zhang_contrastive_learning} improves performance in weakly supervised (no bounding box annotations) settings through contrastive learning. \citet{gigapixel} provides a new dataset for higher resolution images at different granularity and bounding box sizes. \citet{lee2024regroundimprovingtextualspatial} has shown improvements in image generation as well.

This visual grounding task can be understood as the inverse problem to image captioning, where one is given an image and need to provide the text portion. In fact, this inverse paradigm has led to better models \cite{wang2023cycleconsistencylearningcaptioninggrounding} involving cyclic updates.

\subsubsection{Answer Grounding}
Rather than fit a bounding box to various objects, Visual Question Answering (VQA) grounding attempts to find specific parts of an image that corresponds with inputted questions rather than prompts \cite{chen2022groundinganswersvisualquestions, Chen_2023_ICCV_vqa}.

\subsection{Action Grounding}
Relying on other types of grounding such as image grounding, action grounding is a term that refers to building a model that is able to take some grounding and relate it to a set of actions. Recent works utilize LLMs in the fields of chat agents, web agents and robotics \cite{zhang2023llavagroundinggroundedvisualchat, cheng-etal-2024-seeclick,zheng2024gpt4visiongeneralistwebagent,Tellex_Kollar_Dickerson_Walter_Banerjee_Teller_Roy_2011_robotic_navigation,wang2023programmaticallygroundedcompositionallygeneralizable} to motivate better actions that are aligned with people's understanding of the world.

Contrary to using image grounding, \citet{kameko-etal-2015-symbol} matches certain states of games to commentary in an attempt to understand how various actions are grounded in language or its symbols. They refer to this type of grounding as ``symbol grounding'' but essentially attempts to relate some action to some other observation.

\subsection{Audio Grounding}
Audio grounding is the task of taking static images and sounds and attempting to identify which parts of the image are correlated with certain parts of audio. For example, \citet{Tian_2021_CVPR_cyclic_audio} attempts to separate images of bands into which instruments produce what kinds of audio.

\subsection{Video Grounding}
Another related grounding task to visual grounding is the idea of video grounding or spatio-temporal grounding. This task is to identify various portions of a video or the entities within them to provide an understanding for a certain prompt \cite{Jiang_Cheng_Liu_Fang_Peng_Liu_2024_comprehensive_visual_grounding}. These different grounding tasks can be split into its own categories defined in the next sections.

\subsubsection{Object Tracking}
Object tracking relies on the idea that given some natural language prompt, to both identify the specific object within the video but also to continuously track it throughout the video or still frames \cite{Zhou_2023_CVPR_Tracking}.

\subsubsection{Natural Language Spatial Video Grounding}
This video grounding task is an extension of classic visual grounding, where the model attempts to set a bounding box for each frame of a video \cite{li-etal-2022-end-to-end-information-tree,ma2020learninggenerategroundedvisual}.

\subsubsection{Temporal Video Grounding}
This video grounding task is to identify the timestamps in which a prompt holds true for a video \cite{li-etal-2024-groundinggpt, NEURIPS2023_how_to_video,Bao_Zheng_Mu_2021_dense_events, chen-etal-2018-temporally}.

\subsubsection{Spatio-temporal Video Grounding}
This video grounding task combines the last two tasks and attempts to identify both the bounding boxes and the timestamps in which a prompt holds true for a video \cite{Wasim_2024_CVPR_DINO,Chen_2024_CVPR_self_supervised_spatio_temporal,jin2022embracingconsistencyonestageapproach}. It can be used within various settings including video entailment which determines whether a prompt holds true for some video \cite{Chen_2021_ICCV_video_entailment}. Similar to image grounding, video grounding can also be used within video generation tools \cite{jeong2024groundavideozeroshotgroundedvideo}.

\subsection{3D Grounding}
Similar to image grounding, 3D grounding adds a dimension and attempts to put bounding boxes around 3D models which are often represented as point clouds. These 3D grounding tasks share similar strategies to image grounding, using captioning tasks to improve performance \cite{Cai_2022_CVPR_3DJCG,NEURIPS2023_exploiting_context_3d, NEURIPS2023_city_refer_3d,wang20233drpnet3drelativepositionaware}. Some papers have even used 2D object representations to improve 3D grounding \cite{Yang_2021_ICCV_sat_2d}, while others have improved 3D visual grounding with reasoning \cite{zhu2024scanreasonempowering3dvisual}.

\subsection{Dialogue Grounding}
This term of ``dialogue grounding'' is loosely defined, usually seen in literature simply as ``grounding''. Within these papers, ``grounding'' refers to the idea of trying to build a common ground of understanding between two or more actors within a conversation. It includes attempting to analyze nonverbal behaviours \cite{nakano-etal-2003-towards,Roque2007ReactingTA_dialogue, liu-etal-2012-towards,shaikh2024groundinggapslanguagemodel}.

\subsection{Markov Logic Networks Grounding}
``Grounding'' in Markov Logic Networks (MLNs) differs significantly from the other senses of the word \cite{NIPS2014_markov_logic_networks}. MLNs refer to a statistical model for probabilistic logic reasoning, where by developing a set of first-order logic rules known as ``grounds'' one is able to form a weighted satisfiability problem with an optimized solution. In particular, grounding within Markov Logic Networks refers to the process of forming the weighted graph \cite{MLN4KB_markov}.

\subsection{Physical Dynamics Grounding}
Attempting to model physical dynamics purely from states and its transitions tend to be difficult, requiring a ton of resources to supervise consecutive particle properties. Instead of requiring this supervision, a new field has emerged to attempt to understand these physical dynamics from visual observations \cite{NEURIPS2024_neuma_material_visual_grounding}. One such application is in fluid dynamics grounding; which attempts to build an understanding of fluid particle systems from sequential visual observations \cite{guan2022neurofluidfluiddynamicsgrounding}. 

\section{How has ``grounding'' evolved over time?}
In this section, we will take each previous section's meaning of ``grounding'' and build an understanding of its trends through observing the co-occurrence over time with other key words for each specific sense.

\subsection{Terminology}

\subsubsection{Visual Grounding}
TODO: Also explore why there are so many different ways to mean visual grounding. Also explore which datasets are most popular within these papers. Also, explore with removing IJCAI, as empirically it seems to have a lot of overlap with other conferences...

\section{Datasets and Methods}

\section{Discussion}

\section{Conclusion}

% \section{Engines}

% To produce a PDF file, pdf\LaTeX{} is strongly recommended (over original \LaTeX{} plus dvips+ps2pdf or dvipdf).
% The style file \texttt{acl.sty} can also be used with
% lua\LaTeX{} and
% Xe\LaTeX{}, which are especially suitable for text in non-Latin scripts.
% The file \texttt{acl\_lualatex.tex} in this repository provides
% an example of how to use \texttt{acl.sty} with either
% lua\LaTeX{} or
% Xe\LaTeX{}.

% \section{Preamble}

% The first line of the file must be
% \begin{quote}
% \begin{verbatim}
% \documentclass[11pt]{article}
% \end{verbatim}
% \end{quote}

% To load the style file in the review version:
% \begin{quote}
% \begin{verbatim}
% \usepackage[review]{acl}
% \end{verbatim}
% \end{quote}
% For the final version, omit the \verb|review| option:
% \begin{quote}
% \begin{verbatim}
% \usepackage{acl}
% \end{verbatim}
% \end{quote}

% To use Times Roman, put the following in the preamble:
% \begin{quote}
% \begin{verbatim}
% \usepackage{times}
% \end{verbatim}
% \end{quote}
% (Alternatives like txfonts or newtx are also acceptable.)

% Please see the \LaTeX{} source of this document for comments on other packages that may be useful.

% Set the title and author using \verb|\title| and \verb|\author|. Within the author list, format multiple authors using \verb|\and| and \verb|\And| and \verb|\AND|; please see the \LaTeX{} source for examples.

% By default, the box containing the title and author names is set to the minimum of 5 cm. If you need more space, include the following in the preamble:
% \begin{quote}
% \begin{verbatim}
% \setlength\titlebox{<dim>}
% \end{verbatim}
% \end{quote}
% where \verb|<dim>| is replaced with a length. Do not set this length smaller than 5 cm.

% \section{Document Body}

% \subsection{Footnotes}

% Footnotes are inserted with the \verb|\footnote| command.\footnote{This is a footnote.}

% \subsection{Tables and figures}

% See Table~\ref{tab:accents} for an example of a table and its caption.
% \textbf{Do not override the default caption sizes.}

% \begin{table}
%   \centering
%   \begin{tabular}{lc}
%     \hline
%     \textbf{Command} & \textbf{Output} \\
%     \hline
%     \verb|{\"a}|     & {\"a}           \\
%     \verb|{\^e}|     & {\^e}           \\
%     \verb|{\`i}|     & {\`i}           \\
%     \verb|{\.I}|     & {\.I}           \\
%     \verb|{\o}|      & {\o}            \\
%     \verb|{\'u}|     & {\'u}           \\
%     \verb|{\aa}|     & {\aa}           \\\hline
%   \end{tabular}
%   \begin{tabular}{lc}
%     \hline
%     \textbf{Command} & \textbf{Output} \\
%     \hline
%     \verb|{\c c}|    & {\c c}          \\
%     \verb|{\u g}|    & {\u g}          \\
%     \verb|{\l}|      & {\l}            \\
%     \verb|{\~n}|     & {\~n}           \\
%     \verb|{\H o}|    & {\H o}          \\
%     \verb|{\v r}|    & {\v r}          \\
%     \verb|{\ss}|     & {\ss}           \\
%     \hline
%   \end{tabular}
%   \caption{Example commands for accented characters, to be used in, \emph{e.g.}, Bib\TeX{} entries.}
%   \label{tab:accents}
% \end{table}

% As much as possible, fonts in figures should conform
% to the document fonts. See Figure~\ref{fig:experiments} for an example of a figure and its caption.

% Using the \verb|graphicx| package graphics files can be included within figure
% environment at an appropriate point within the text.
% The \verb|graphicx| package supports various optional arguments to control the
% appearance of the figure.
% You must include it explicitly in the \LaTeX{} preamble (after the
% \verb|\documentclass| declaration and before \verb|\begin{document}|) using
% \verb|\usepackage{graphicx}|.

% \begin{figure}[t]
%   \includegraphics[width=\columnwidth]{example-image-golden}
%   \caption{A figure with a caption that runs for more than one line.
%     Example image is usually available through the \texttt{mwe} package
%     without even mentioning it in the preamble.}
%   \label{fig:experiments}
% \end{figure}

% \begin{figure*}[t]
%   \includegraphics[width=0.48\linewidth]{example-image-a} \hfill
%   \includegraphics[width=0.48\linewidth]{example-image-b}
%   \caption {A minimal working example to demonstrate how to place
%     two images side-by-side.}
% \end{figure*}

% \subsection{Hyperlinks}

% Users of older versions of \LaTeX{} may encounter the following error during compilation:
% \begin{quote}
% \verb|\pdfendlink| ended up in different nesting level than \verb|\pdfstartlink|.
% \end{quote}
% This happens when pdf\LaTeX{} is used and a citation splits across a page boundary. The best way to fix this is to upgrade \LaTeX{} to 2018-12-01 or later.

% \subsection{Citations}

% \begin{table*}
%   \centering
%   \begin{tabular}{lll}
%     \hline
%     \textbf{Output}           & \textbf{natbib command} & \textbf{ACL only command} \\
%     \hline
%     \citep{Gusfield:97}       & \verb|\citep|           &                           \\
%     \citealp{Gusfield:97}     & \verb|\citealp|         &                           \\
%     \citet{Gusfield:97}       & \verb|\citet|           &                           \\
%     \citeyearpar{Gusfield:97} & \verb|\citeyearpar|     &                           \\
%     \citeposs{Gusfield:97}    &                         & \verb|\citeposs|          \\
%     \hline
%   \end{tabular}
%   \caption{\label{citation-guide}
%     Citation commands supported by the style file.
%     The style is based on the natbib package and supports all natbib citation commands.
%     It also supports commands defined in previous ACL style files for compatibility.
%   }
% \end{table*}

% Table~\ref{citation-guide} shows the syntax supported by the style files.
% We encourage you to use the natbib styles.
% You can use the command \verb|\citet| (cite in text) to get ``author (year)'' citations, like this citation to a paper by \citet{Gusfield:97}.
% You can use the command \verb|\citep| (cite in parentheses) to get ``(author, year)'' citations \citep{Gusfield:97}.
% You can use the command \verb|\citealp| (alternative cite without parentheses) to get ``author, year'' citations, which is useful for using citations within parentheses (e.g. \citealp{Gusfield:97}).

% A possessive citation can be made with the command \verb|\citeposs|.
% This is not a standard natbib command, so it is generally not compatible
% with other style files.

% \subsection{References}

% \nocite{Ando2005,andrew2007scalable,rasooli-tetrault-2015}

% The \LaTeX{} and Bib\TeX{} style files provided roughly follow the American Psychological Association format.
% If your own bib file is named \texttt{custom.bib}, then placing the following before any appendices in your \LaTeX{} file will generate the references section for you:
% \begin{quote}
% \begin{verbatim}
% \bibliography{custom}
% \end{verbatim}
% \end{quote}

% You can obtain the complete ACL Anthology as a Bib\TeX{} file from \url{https://aclweb.org/anthology/anthology.bib.gz}.
% To include both the Anthology and your own .bib file, use the following instead of the above.
% \begin{quote}
% \begin{verbatim}
% \bibliography{anthology,custom}
% \end{verbatim}
% \end{quote}

% Please see Section~\ref{sec:bibtex} for information on preparing Bib\TeX{} files.

% \subsection{Equations}

% An example equation is shown below:
% \begin{equation}
%   \label{eq:example}
%   A = \pi r^2
% \end{equation}

% Labels for equation numbers, sections, subsections, figures and tables
% are all defined with the \verb|\label{label}| command and cross references
% to them are made with the \verb|\ref{label}| command.

% This an example cross-reference to Equation~\ref{eq:example}.

% \subsection{Appendices}

% Use \verb|\appendix| before any appendix section to switch the section numbering over to letters. See Appendix~\ref{sec:appendix} for an example.

% \section{Bib\TeX{} Files}
% \label{sec:bibtex}

% Unicode cannot be used in Bib\TeX{} entries, and some ways of typing special characters can disrupt Bib\TeX's alphabetization. The recommended way of typing special characters is shown in Table~\ref{tab:accents}.

% Please ensure that Bib\TeX{} records contain DOIs or URLs when possible, and for all the ACL materials that you reference.
% Use the \verb|doi| field for DOIs and the \verb|url| field for URLs.
% If a Bib\TeX{} entry has a URL or DOI field, the paper title in the references section will appear as a hyperlink to the paper, using the hyperref \LaTeX{} package.

\section*{Limitations}

% Since December 2023, a "Limitations" section has been required for all papers submitted to ACL Rolling Review (ARR). This section should be placed at the end of the paper, before the references. The "Limitations" section (along with, optionally, a section for ethical considerations) may be up to one page and will not count toward the final page limit. Note that these files may be used by venues that do not rely on ARR so it is recommended to verify the requirement of a "Limitations" section and other criteria with the venue in question.

% \section*{Acknowledgments}

% This document has been adapted
% by Steven Bethard, Ryan Cotterell and Rui Yan
% from the instructions for earlier ACL and NAACL proceedings, including those for
% ACL 2019 by Douwe Kiela and Ivan Vuli\'{c},
% NAACL 2019 by Stephanie Lukin and Alla Roskovskaya,
% ACL 2018 by Shay Cohen, Kevin Gimpel, and Wei Lu,
% NAACL 2018 by Margaret Mitchell and Stephanie Lukin,
% Bib\TeX{} suggestions for (NA)ACL 2017/2018 from Jason Eisner,
% ACL 2017 by Dan Gildea and Min-Yen Kan,
% NAACL 2017 by Margaret Mitchell,
% ACL 2012 by Maggie Li and Michael White,
% ACL 2010 by Jing-Shin Chang and Philipp Koehn,
% ACL 2008 by Johanna D. Moore, Simone Teufel, James Allan, and Sadaoki Furui,
% ACL 2005 by Hwee Tou Ng and Kemal Oflazer,
% ACL 2002 by Eugene Charniak and Dekang Lin,
% and earlier ACL and EACL formats written by several people, including
% John Chen, Henry S. Thompson and Donald Walker.
% Additional elements were taken from the formatting instructions of the \emph{International Joint Conference on Artificial Intelligence} and the \emph{Conference on Computer Vision and Pattern Recognition}.

% Bibliography entries for the entire Anthology, followed by custom entries
%\bibliography{anthology,custom}
% Custom bibliography entries only
\bibliography{custom}

% \appendix

% \section{Example Appendix}
% \label{sec:appendix}

% This is an appendix.

\end{document}
