\documentclass[12pt, letterpaper]{article}
% Packages
\usepackage[left=1in, right=1in, top=1in, bottom=1.2in,footskip=0.3in]{geometry}
\usepackage[utf8]{inputenc}
\usepackage[dvipsnames]{xcolor}
\usepackage[ruled,vlined,noresetcount]{algorithm2e}
\usepackage{amsmath}
\usepackage{amsfonts}
\usepackage{amssymb}
\usepackage{amsthm}
\usepackage{arydshln}
\usepackage{booktabs}
\usepackage{algpseudocode}
\usepackage{bbm}
\usepackage{bm}
\usepackage{enumitem}
\usepackage{float}
\usepackage[T1]{fontenc}
\usepackage{footnote}
\usepackage[linguistics]{forest}
\usepackage{graphicx}
\usepackage[hidelinks,backref=page,colorlinks=true,linkcolor=black,citecolor=black,urlcolor=black]{hyperref}
\usepackage{listings}
\usepackage{mathpazo}
\usepackage{mathtools}
\usepackage{microtype}
\usepackage{multirow}
\usepackage[numbers,authoryear]{natbib}
\usepackage{subcaption}
\usepackage{tablefootnote}
\usepackage{titlesec}
\usepackage{tikz}
\usepackage{tikz-dependency}
\usepackage{url}
\usepackage{xspace}
\usepackage[capitalize,noabbrev]{cleveref}
\crefname{chapter}{Chapter}{Chapters}
\crefname{section}{\S}{\S\S}
\Crefname{section}{\S}{\S\S}
\crefname{table}{Table}{Tables}
\crefname{figure}{Figure}{Figures}
\crefname{algorithm}{Algorithm}{}
\crefname{equation}{Eq.}{}
\crefname{appendix}{Appendix}{}
\crefformat{section}{\S#2#1#3}
\crefname{lemma}{Lemma}{Lemmas}
\crefname{proposition}{Proposition}{Propositions}
\crefname{definition}{Definition}{Definitions}
\crefname{corollary}{Corollary}{Corollaries}
\crefname{example}{Example}{Examples}
\crefname{problem}{Problem}{Problems}
\crefname{remark}{Remark}{Remarks}

\setlist[itemize]{itemsep=1pt, topsep=1pt}
\setlist[enumerate]{itemsep=1pt, topsep=1pt}

% ----------------
% Notation
\DeclareMathOperator*{\argmax}{arg\,max}
\DeclareMathOperator*{\argmin}{arg\,min}
\newcommand{\stderr}{$\pm$}

% ----------------
% Commands
\newcommand{\subsubsubsection}[1]{\vspace{0.1in}\noindent{\bf #1:}}
\newcommand{\interalia}[1]{\cite[][\textit{inter alia}]{#1}}
\makeatletter
\def\adl@drawiv#1#2#3{%
        \hskip.5\tabcolsep
        \xleaders#3{#2.5\@tempdimb #1{1}#2.5\@tempdimb}%
                #2\z@ plus1fil minus1fil\relax
        \hskip.5\tabcolsep}
\newcommand{\cdashlinelr}[1]{%
  \noalign{\vskip\aboverulesep
           \global\let\@dashdrawstore\adl@draw
           \global\let\adl@draw\adl@drawiv}
  \cdashline{#1}
  \noalign{\global\let\adl@draw\@dashdrawstore
           \vskip\belowrulesep}}
\makeatother

% ----------------
% Formatting
\setlength{\parskip}{0.5ex}
\titlespacing*{\paragraph}{0pt}{1ex plus 0.5ex minus 0.2ex}{0.5ex}
\renewcommand{\baselinestretch}{1.241}

% ----------------
% User Commands
\newcommand{\task}[2]{\textcolor{blue}{Task #1: #2}}

% ----------------
% Tikz Commands
\usetikzlibrary{decorations.pathreplacing}


% ----------------
% Math Commands
\newtheorem{assumption}{Assumption}
\newtheorem{theorem}{Theorem}
\theoremstyle{definition}
\newtheorem{corollary}[theorem]{Corollary}
\newtheorem{example}[theorem]{Example}
\newtheorem{proposition}[theorem]{Proposition}
\newtheorem{definition}[theorem]{Definition}
\newtheorem{problem}[theorem]{Problem}
\DeclareMathOperator*{\suchthat}{\textit{s.t.}}


\begin{document}

\title{CS 784 Final Project: Midterm Check-in}
\author{Eric Huang}
\maketitle

This is the midterm check-in for the final project. For my word, I have chosen the word `grounding'.

\section{What I've done}
\subsection{Meta-analysis}
Before choosing a word, I decided to first perform some meta-analysis to understand which papers were more important and how overloaded and saturated the term was. To understand which papers were more important, I took the top-p papers (i.e. top papers until we meet some percentage $p$ of the total word count across the entire dataset) based on the relative frequency, i.e. the count of a certain word compared to its total count of words. I performed this per split of the dataset, not mixing up conferences. Note that I simply used the pure substring and not word based on boundaries when calculating this; this ensures that plurality will not cause any miscounts.\\

After performing this analysis, with $p = 0.1$, I decided to choose the word `grounding' which had 52 papers total, making it possible to read through all of these. Across all these papers, the highest word frequency was 2.16\% found in the ECCV conference for a Llava grounded visual chat \cite{zhang2023llavagroundinggroundedvisualchat}.

\subsection{Paper Reading}
I have read a small subset of the top-p papers described above on uncovering the following meanings of the word `grounding':
\begin{enumerate}
    \item Image grounding: A fine-grained understanding of images, including specific regions and alignment \cite{zhang2023llavagroundinggroundedvisualchat,li-etal-2024-groundinggpt}.
    \item Video Grounding: Focuses on identifying and localizing specific moments in the video based on descriptions \cite{li-etal-2024-groundinggpt}.
    \item Spatio-Temporal Video Grounding (might just be a very similar task as video grounding): Focuses on identifying a certain object within videos based on query sentences \cite{wasim2024videogroundingdinoopenvocabularyspatiotemporalvideo}.
    \item Grounding box: An annotated box utilized within image datasets to provide examples of what the ground truth is. Usually used in instances of trying to identify certain objects \cite{li-etal-2024-groundinggpt}.
\end{enumerate}
Generally, grounding means developing an understanding of the composition of the input. The term is overloaded in terms of the precise composition based on the modality.\\

I have also discovered that the terminology of `understanding' is tied to `grounding'. Usually, by improving `grounding', i.e. figuring out how the input works, one can improve the results and reduce hallucinations.

\section{What I plan to do next}
I need to continue reading through the list of 52 papers I have compiled, and bin them into the different ways that grounding is used. Based on these different meanings, I should rerun some analysis to try and see if I can analyze how many papers use grounding in their different senses.\\

I should also do some contrastive analysis and look at the bottom-p and middle-p papers to understand whether or not they use grounding within a separate context. Another simple extension is to add in the word grounded, as it seems as though that is another commonly used term in lieu of grounding.\\

Finally, I plan on building an understanding of how people tackle grounding tasks within multimodal settings alongside common benchmarks used. I will conduct an analysis on the papers that include sufficient `grounding' instances and analyze the proportion of them that use certain public benchmarks. I plan to explore the benchmark papers themselves and analyze the citation graph to understand how everything works together.

\bibliographystyle{plain}
\bibliography{references.bib}

\end{document}
