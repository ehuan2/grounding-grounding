\documentclass[12pt, letterpaper]{article}
\input{macros.tex}

\begin{document}

\title{CS 784 Final Project: Midterm Check-in}
\author{Eric Huang}
\maketitle

This is the midterm check-in for the final project. For my word, I have chosen the word `grounding'.

\section{What I've done}
\subsection{Meta-analysis}
Before choosing a word, I decided to first perform some meta-analysis to understand which papers were more important and how overloaded and saturated the term was. To understand which papers were more important, I took the top-p papers (i.e. top papers until we meet some percentage $p$ of the total word count across the entire dataset) based on the relative frequency, i.e. the count of a certain word compared to its total count of words. I performed this per split of the dataset, not mixing up conferences. Note that I simply used the pure substring and not word based on boundaries when calculating this; this ensures that plurality will not cause any miscounts.\\

After performing this analysis, with $p = 0.1$, I decided to choose the word `grounding' which had 52 papers total, making it possible to read through all of these. Across all these papers, the highest word frequency was 2.16\% found in the ECCV conference for a Llava grounded visual chat \cite{zhang2023llavagroundinggroundedvisualchat}.

\subsection{Paper Reading}
I have read a small subset of the top-p papers described above on uncovering the following meanings of the word `grounding':
\begin{enumerate}
    \item Image grounding: A fine-grained understanding of images, including specific regions and alignment \cite{zhang2023llavagroundinggroundedvisualchat,li-etal-2024-groundinggpt}.
    \item Video Grounding: Focuses on identifying and localizing specific moments in the video based on descriptions \cite{li-etal-2024-groundinggpt}.
    \item Spatio-Temporal Video Grounding (might just be a very similar task as video grounding): Focuses on identifying a certain object within videos based on query sentences \cite{wasim2024videogroundingdinoopenvocabularyspatiotemporalvideo}.
    \item Grounding box: An annotated box utilized within image datasets to provide examples of what the ground truth is. Usually used in instances of trying to identify certain objects \cite{li-etal-2024-groundinggpt}.
\end{enumerate}
Generally, grounding means developing an understanding of the composition of the input. The term is overloaded in terms of the precise composition based on the modality.\\

I have also discovered that the terminology of `understanding' is tied to `grounding'. Usually, by improving `grounding', i.e. figuring out how the input works, one can improve the results and reduce hallucinations.

\section{What I plan to do next}
I need to continue reading through the list of 52 papers I have compiled, and bin them into the different ways that grounding is used. Based on these different meanings, I should rerun some analysis to try and see if I can analyze how many papers use grounding in their different senses.\\

I should also do some contrastive analysis and look at the bottom-p and middle-p papers to understand whether or not they use grounding within a separate context. Another simple extension is to add in the word grounded, as it seems as though that is another commonly used term in lieu of grounding.\\

Finally, I plan on building an understanding of how people tackle grounding tasks within multimodal settings alongside common benchmarks used. I will conduct an analysis on the papers that include sufficient `grounding' instances and analyze the proportion of them that use certain public benchmarks. I plan to explore the benchmark papers themselves and analyze the citation graph to understand how everything works together.

\bibliographystyle{plain}
\bibliography{references.bib}

\end{document}
